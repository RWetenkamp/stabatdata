\documentclass[11pt,a4paper,twocolumn]{scrartcl}
\usepackage[utf8]{inputenc}
\usepackage[english]{babel}
\usepackage{amsmath}
\usepackage{amsfonts}
\usepackage{amssymb}
\usepackage{makeidx}
\usepackage{hyperref}
\usepackage[left=2cm,right=2cm,top=2cm,bottom=2cm]{geometry}
\title{On drafting a self-deleting data format}
\subtitle{A concept for ensuring data privacy for users and consumers}
\author{Roman Wetenkamp \\\url{info@rwetenkamp.de}}
\begin{document}
\maketitle
\tableofcontents
\section{Introduction}
\paragraph{Ensure privacy} It is almost impossible to control whether the privacy of your data is guaranteed permanently. Cyber attacks, negligant dealing with data or illegal actions - We need to trust the companies working with our data, hopefully they know what they do - but is there a way to ensure that our data is kept secret?
\paragraph{Change of roles}
If we find a way to format our information and data in a way it is deleting itself triggered by a specified incident like a point in time, a crossed border or a modification, the control of our privacy become our job and gives us more control in defining who and how our data is used and transmitted.
\paragraph{{\glqq}They do not know what they do{\grqq}} Knowledge about privacy rights, ways to protect data and basic strategies of formatting and storing data in secure way is spread far to less in society. This will cause several problems and the datas owner is not able to keep its data safe anymore. While education is of course on part of the solution, the way we work with data is another.
\\\\
These are the reasons why it is necessary to think about a technological solution solving this problems.
\section{Definitions}
\paragraph{data format} The way data is formatted and transfered through a network or software solution, e.g. as a file type like XML, JSON, CSV, SQL or as text / binaries etc.
\paragraph{self-deletion} Selected and specified parts of a dataset transform through a cryprographic operation into a state where absolutely noone is able to identify the original data.
\paragraph{privacy laws} Laws made by parliaments to control and ensure the safety of citizens data.{\footnote{GER: DSGVO -- Datenschutzgrundverordnung}}
\end{document}